\documentclass[12pt]{article}
\usepackage{graphicx}
%\setlength{\parindent}{0in}

% General Latex  --------------------------------------------------
\def\beq{\begin{equation}}
\def\eeq{\end{equation}}
\def\beqar{\begin{eqnarray}}
\def\eeqar{\end{eqnarray}}
\def\nn{\nonumber}
\def\ol{\overline}
\def\para{\parallel}

% Operators  ------------------------------------------------------
\newcommand{\diff}[2]{\frac{d#1}{d#2}}
\newcommand{\diffs}[2]{\frac{d^2#1}{d#2^2}}
\newcommand{\pdiff}[2]{\frac{\partial#1}{\partial#2}}
\newcommand{\pdiffs}[2]{\frac{\partial^2#1}{\partial#2^2}}
\newcommand{\pdiffxy}[3]{\frac{\partial^2#1}{\partial#2 \partial#3}}
\newcommand{\pdt}{\partial_t}
\newcommand{\pdr}{\partial_r}
\newcommand{\pdth}{\partial_\theta}
\newcommand{\pdrr}{\partial^2_r}

\newcommand{\enum}[2]{{#1}\times10^{#2}} % 4.2x10^{3} = \enum{4.2}{3}

\newcommand{\vect}[1]{{\bf #1}}
%\newcommand{\vect}{\overrightarrow}
%\newcommand{\vect}{\vec}
\def\div{\nabla\cdot}
\def\grad{\nabla}
\def\curl{\nabla\times}
\newcommand{\gradpar}{\grad_\parallel}
\newcommand{\gradperp}{\grad_\perp}
\newcommand{\gradr}{\grad_r}
\newcommand{\defeq}{\ensuremath{\stackrel{\text{\tiny def}}{=}}}

\newcommand{\savg}[1]{\left<{#1}\right>}
\newcommand{\vavg}[1]{\left<{#1}\right>_V}
\newcommand{\thavg}[1]{\left<{#1}\right>_\theta}

% Variable names  -------------------------------------------------
\newcommand{\vpar} {v_\parallel}
\newcommand{\Apar} {A_\parallel}
\newcommand{\jpar} {j_\parallel}
\newcommand{\kpar} {k_\parallel}
\newcommand{\kperp} {k_\perp }
\newcommand{\vperp} {v_\perp }
\newcommand{\kthe}{k_\theta}

\newcommand{\Evec}{\ensuremath{\boldsymbol{{\rm E}}}}
\newcommand{\Bvec}{\ensuremath{\boldsymbol{{\rm B}}}}
\newcommand{\Jvec}{\ensuremath{\boldsymbol{{\rm J}}}}
\newcommand{\Fvec}{\ensuremath{\boldsymbol{{\rm F}}}}
\newcommand{\fvec}{\ensuremath{\boldsymbol{{\rm f}}}}
\newcommand{\vE}{\ensuremath{\boldsymbol{{\rm v}_{E}}}}
\newcommand{\bo}{\ensuremath{\boldsymbol{{\rm b}_0}}}
\newcommand{\bvec}{\ensuremath{\boldsymbol{{\rm b}}}}
\newcommand{\xvec}{\ensuremath{\boldsymbol{{\rm x}}}}
\newcommand{\yvec}{\ensuremath{\boldsymbol{{\rm y}}}}
\newcommand{\zvec}{\ensuremath{\boldsymbol{{\rm z}}}}
\newcommand{\vvec}{\ensuremath{\boldsymbol{{\rm v}}}}
\newcommand{\jvec}{\ensuremath{\boldsymbol{{\rm j}}}}

\newcommand{\bxgp}{\bvec\times\gradperp}

\newcommand{\vve}{\ensuremath{\boldsymbol{{\rm v}}_{e}}}
\newcommand{\vvi}{\ensuremath{\boldsymbol{{\rm v}}_{i}}}
\newcommand{\vpe}{v_{\parallel e}}
\newcommand{\vpi}{v_{\parallel i}}
\newcommand{\vvE}{\ensuremath{\boldsymbol{{\rm v}}_{E}}}
\newcommand{\vvD}{\ensuremath{\boldsymbol{{\rm v}}_{D}}}

\newcommand{\nuei}{\nu_{ei}}
\newcommand{\nuii}{\nu_{ii}}
\newcommand{\nue}{\nu_{e}}
\newcommand{\nuen}{\nu_{en}}
\newcommand{\nuin}{\nu_{in}}
\newcommand{\kpe}{\kappa_{\parallel e}}

\newcommand{\rs}{\rho_{s}}
\newcommand{\ri}{\rho_{i}}
\newcommand{\wci}{\Omega_{i}}
\newcommand{\wcix}{\Omega_{ix}}
\newcommand{\wce}{\Omega_{e}}
\newcommand{\tomega}{\tilde\omega}
\newcommand{\Isat}{I_{\rm sat}}
\newcommand{\fmie}{\frac{m_i}{m_e}}
\newcommand{\fmei}{\frac{m_e}{m_i}}


% Often used dimensions
\newcommand{\cm}{\rm cm}
\newcommand{\mm}{\rm mm}
\newcommand{\cmn}{{\rm cm}^{-3}}
\newcommand{\mn}{{\rm m}^{-3}}
\newcommand{\eV}{\rm eV}
\newcommand{\G}{\rm G}
\newcommand{\T}{\rm T}



\begin{document}


Response to the referee's comments \\ \\ 

We thank the referee for the thoughtful review of our paper and noting that we present a ``thorough and compelling analysis'' of a nonlinear instability. We believe that the referee's comments will help
improve our paper by making us provide some important details that we had left out at first. The following are our responses to the referee's comments. \\

Comment 1:   On page 3, the paper states that the “nonlinear drift wave instability dominates when
the fluctuation amplitude becomes large enough”. Can the authors provide an estimate of the threshold? \\

Response: The nonlinear instability reaches an equal drive level as the linear instability when the RMS fluctuation level is about 2-3$\%$ of the equilibrium, making it about $1/5$ 
of the fluctuation value during saturation. At this point, there is nearly equal energy in $n=0$ and $n= \pm 1$ modes. 
As the fluctuation amplitude passes this value, the nonlinear instability becomes dominant. We agree with the referee that this is important information and
have added some text on pages 3 and 9 stating these points. \\


Comment 2:    Dirichlet boundary conditions are used for the fluctuations at the radial boundaries r = 12 and 40 cm. The authors should indicate how the boundary values are chosen. \\

Response: We have indicated in the text that the radial boundary locations used in the simulation correspond to the locations in the experiment where the fluctuations die away to only
a few percent of the equilibrium values as seen in Figure 2c. This should justify use of Dirichlet boundary conditions at these points. \\

Comment 3:  In Figure 2, and other figures, the traces are hard to distinguish without color. Unless all figures will be rendered in color in the print version, the traces should be better
distinguished. \\

Response: We do not wish for the figures to be rendered in color in the print version, so we have added different line styles or markers on all line plots so that they can be 
distinguished without color. \\

Comment 4:   Figure 2 gives the only comparison between experiment and simulation. The n = 0
fluctuations are said to dominate at various places in the paper, both for simulation and
experiment. Since this is an important point, it would be helpful if the n spectrum for a
fixed value of m were shown as a part (d) of the figure, again with numerical and
experimental traces. \\

Response: The referee is right to point out that a comparison of the axial wavenumber spectrum between simulation and experiment would significantly enhance our evidence for our conclusion.
Unfortunately, we do not believe that we can accurately measure the axial wavenumber $n$ spectrum in the experiment. In order to experimentally measure a wavenumber spectrum, one must
use multiple probes, say two of them, and observe the correlation between the two. To get the axial wavenumber spectrum, one must align the two probes along a magnetic field line.
This is extremely difficult to do in practice, and any misalignment will cause the correlation characteristics to be dominated by the perpendicular misalignment since the perpendicular
wavelength is so much smaller than the parallel one. We realize that such a measurement would provide the greatest
evidence for the presence of the nonlinear instability in the experiment and it is a goal of ours to find a way to do this in the future. 
But since such a measurement is not likely to be accurate at the moment, we have had to rely on more indirect evidence, such as that discussed in Section V. \\
   

Comment 5: In Fig. 2 (c) the simulation and experiment track each other reasonably well for r > 20
cm. For r < 20 cm the two traces have different behavior. In light of the paper’s
statement that “the simulation reproduces these characteristics of experimental
measurements with rather good qualitative and quantitative accuracy”, the differences for
r < 20 cm should be explained. \\

Response: We acknowledge that there is an area between 15-20 cm where the fluctuation levels disagree at a higher level than elsewhere. We believe that this discrepency is due to the smoothed
profiles that we use in the simulation. It is clear that at other radii, the fluctuation level of the experiment is bumpier than that of the simulation, and this is also a result
of a bumpier experimental profile. We did not work on fine tuning our simulation profiles to achieve better validation because validation wasn't the primary intent of this study,
and we felt that the agreement between simulation and experiment was good enough to proceed with the energetics analysis. \\

Comment 6:      tried to determine the magnitude of the zonal flow, zonal density, and zonal
temperature from Fig. 4, but found it difficult to reliably extract that information from the
color scale for the narrow slice that represents m = 0. This is especially true if one is
looking at a black and white rendition of the figure. On page 18, where zonal flows are
discussed, it would be helpful to have an idea how large the zonal flow is. \\

Response: The zonal density and temperature are identically zero due to the density and temperature sources which subtract out the $m=0$ component of these fluctuations. The zonal flow has an energy
of $1.4 \times 10^{-4}$. We have indicated these points in the caption of Figure 4. \\

Comment 7:   The last sentence of page 11 could profitably be split into two easier-to-digest sentences. \\

Response: The original sentence has been split into two sentences. \\

Comment 8:   On page 12 the paper states that the dissipation from the source is not shown in the
figure but is “rather large”. This magnitude of the dissipation should be quantified. \\

Response: We have indicated the magnitude of the source in the text. \\

Comment 9:   In the sentence that starts at the bottom of page 12 it would be helpful to indicate that
small value of Ct is something that is demonstrated in part (d) of the figure. \\

Response: This point is made two paragraphs later during the discussion of Figure 6d. But we have added a small clarification of this point near the end of the paragraph that discusses Figure 6d. \\

Comment 10:   Although it is said nicely somewhere earlier in the paper, it would be appropriate to
restate in the conclusions that the analysis of the paper shows that n = 0 is excited by
nonlinear instability and not conservative transfer. \\

Response: This point is now restated in the conclusion. \\

Comment 11:  In the list of references, the same convention for author’s initials as used in the cited
papers should be followed, i.e., if the article cited uses two initials for an author, both
initials should be included. \\

Response: We have fixed and reviewed the author initials in all citations. \\

Comment 12: In references 28 and 29 the redundant “and” should be removed. \\

Response: We removed the redundant ``and''s. \\


Additional author comment 1: After submission, we discovered a small error in our ccomparison of the amplitudes of the power spectra in Fig 2a and Fig 10. 
We have corrected the error by using Parseval's Theorem to correctly weight the spectra. The new figures do not change any of the conclusions of the paper and no text needed to be changed as a result. \\

Additional author comment 2: After submission, we had a conversation with someone who read the paper on the normalizations that we used in our energy equation. As a result of this discussion,
we have added a few lines of text to the end of Section III which clarifies our position in using the normalizations that we did. \\

Sincerely,
Brett Friedman

\end{document}
